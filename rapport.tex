\documentclass[a4paper,10pt]{article}

\usepackage{fullpage}%
\usepackage[T1]{fontenc}%
\usepackage[utf8]{inputenc}%
\usepackage[main=francais,english]{babel}%
\usepackage{graphicx}%
\usepackage{url}%
\usepackage{abstract}%
\usepackage{listings}%
\usepackage{tikz}%
\usepackage{amsmath}%
\usepackage{amssymb}%
\usepackage{amsfonts}%
\usepackage{mathtools}%
\usepackage{amsthm}%
\usepackage{array}%
\usepackage{csquotes}%
\usepackage{mathpazo}%
\usepackage{subfig}%
\usepackage[backend=biber]{biblatex}%

\DeclarePairedDelimiter\ceil{\lceil}{\rceil}%
\DeclarePairedDelimiter\floor{\lfloor}{\rfloor}%

\renewcommand\qedsymbol{$\blacksquare$}%

\theoremstyle{definition}

\newtheorem{defi}{Définition}
\newtheorem*{defi*}{Définition}
\newtheorem{theo}{Théorème}
\newtheorem*{theo*}{Théorème}
\newtheorem{exem}{Exemple}
\newtheorem*{exem*}{Exemple}
\newtheorem{prop}{Proposition}
\newtheorem*{prop*}{Proposition}
\newtheorem{coro}{Corollaire}
\newtheorem*{coro*}{Corollaire}
\newtheorem{lemm}{Lemme}
\newtheorem*{lemm*}{Lemme}

\bibliography{papers}% The name of your .bib file

\parskip=0.5\baselineskip

\lstset{%
	basicstyle= \sffamily,%
	columns=fullflexible,%
	frame=lb,%
	frameround=fftf,%
	language=C,%
	numbers=left,%
}%

%
\begin{document}
%
\title{PROG2 : Projet PONG}
%
\author{Tom Bachard\and Guillaume Barbier\and Victor Careil}
%
\date{\today}
%
\maketitle
%
%

\begin{description}
  \item[Classification ACM] D.m
  \item[Mots-clés] Programmation; Java; JavaFX; Programmation événementielle; PONG
\end{description}

%
\begin{abstract}
Le paradigme de programmation événementielle est très souvent utilisé lorsqu'un utilisateur interagit de manière intermittente avec l'application. Grâce à la bibliothèque JavaFX de Java (anciennement Swing), nous mettons au point un moteur de jeu permettant de jouer au très célèbre jeu PONG.
\end{abstract}
%

%
\section{Introduction}
%

  \subsection{PONG}
  	%
	PONG est l'un des premiers jeux vidéos commercialisés. Inventé en 1972 par Nolan Bushnell et Allan Alcorn, ce jeu propose de faire s'affronter tantôt deux joueurs, tantôt un joueur et un ordinateur, dans un match de tennis de table vu du dessus. C'est le premier jeu qui devient populaire, en attestent les 8000 bornes d'arcade vendues par \emph{Atari} l'année de son invention.
	
	Lors d'un affrontement joueur contre joueur, le jeu doit être capable de gérer les déplacements des deux raquettes en même temps, tout en calculant la trajectoire de la balle, afin que cette dernière soit cohérente et permette une bonne expérience de jeu. Pour cela, nous développons notre PONG en Java, en nous servant de la bibliothèque JavaFX (anciennement Swing), spécialisée dans le traitement d'événements interactifs concurrents.
	
  \subsection{Programmation événementielle}
  	%
	La programmation événementielle est un paradigme de programmation qui, par opposition à la programmation séquentielle, s'appuie sur les événements et leur réactions qu'elle reçoit. Ce qu'on définit comme \og événement \fg{} est très large, en effet, sont considérés comme tels les actions suivantes:
	\begin{itemize}
		\item l'ajout d'élément dans une liste ;
		\item le changement d'état d'un booléen ;
		\item le clic d'une souris ;
		\item la pression d'une touche de clavier\dots
	\end{itemize} 
	
	On remarque que ces points, et notamment les deux derniers, sont très intéressants dans la réalisation d'un jeu tel que PONG. En effet, considérer les positions des différentes raquettes, ainsi que la balle, comme des événements permet une implantation simple et élégante de ce jeu.
	
	Enfin, il est à noté que la programmation événementielle est généralement utilisée avec des langages de haut niveau, ce qui justifie notre utilisation de Java.
	
\section{Concepts généraux}
%

    \subsection{Les timelines !}
    %
    Écrire un truc sur ces bidules. :\^{})
    
    \subsection{Utilisation des \emph{bindings}}
    %
    Afin de répondre au mieux au problème posé, nous avons décidé d'utiliser les \emph{bidings} proposé par \lstinline{JavaFX}. En effet, ces derniers permettent de lier nos objets, et surtout leur propriétés, aux événements extérieurs, notamment liés aux joueurs. Sans entrer directement dans les détails, on voit bien que l'on pourra, entre autres, lier la position des raquettes en fonction des touches pressées facilement grâce à ce système.
    
    \subsection{Séparation du rendu et des calculs}
    %
    Dans l'optique de rendre notre code solide et modulable, nous appliquerons le principe de \og séparation des deux mondes \fg{}. En effet, le monde des calculs mathématiques, trajectoires, rebonds, \dots, et celui du rendu des figures à l'écran, balle, raquettes, \dots, ne doivent pas se croiser. Les classes graphiques ne doivent rien calculer, cela permet de changer radicalement la manière de calcul sans avoir à toucher à ces classes. On pourra même attester de cette robustesse dans notre code puisqu'après la seconde démonstration, nos collisions ont été entièrement refaites, sans avoir à toucher à quoi que se soit dans nos classes graphiques.
    
\section{Implémentation de notre PONG}
%

Lors de la réalisation du jeu, nous avons séparé les classes en deux catégories, comme expliqué précédemment. D'une part, nos classes de calculs, vivants dans un monde purement mathématique, et d'autre part nos classes graphiques, qui appellent les fonctions des premiers classes afin de dessiner les objets aux bons endroits.

    \subsection{Classes du moteur}
    %
    Nous avons choisi de créer un petit moteur de jeu lors de ce projet, principalement de rendre les collisions des balles avec les murs ou raquettes. On pourra noter dans le sous groupe de fichiers \emph{core} des énumérations et des interfaces, tant pour la lisibilité du code que pour sa modularité.
    	
	\subsubsection{Classe \emph{Vector2D}}
	%
	Ce classe contient une classe abstraite de vecteur en deux dimensions, utiles pour tous nos calculs. On y retrouve les fonctions classiques d'addition, multiplication, de norme, mais aussi des plus spécifiques à notre jeu telles que des rotations ou de calcul de normal.
	
	\subsubsection{Classe \emph{Engine}}
	%
	Le fichier \emph{Engine} est la pièce principale du moteur. En effet, c'est lui qui contient l'entièreté des constantes liées au jeu. De plus, des méthodes classiques aux moteurs de jeu nous permettent de gérer facilement le lancement du jeu, son arrêt, les conditions de victoires, \emph{etc}, dans les autres fichiers.
	
	C'est aussi ce fichier qui gère la scène qui contiendra les objets que l'on générera et déplacera lors du jeu. On notera que les différents \emph{bindings} sont créés à ce moment du processus.
	
	Enfin, c'est au travers de ce classe que sont gérés les affichages des scores et autres textes durant la partie.
	
        \subsubsection{Classe \emph{World}}
        %
        Ce classe permet de créer la zone de jeu, dont les bordures de la fenêtre, et d'y ajouter les collisions avec les différents éléments présents dans la zone de jeu. On voit aussi qu'il change la direction de la balle en présence d'une collision.
        
        \subsubsection{Classe \emph{Borders}}
        %
        L'unique but de ce classe est de calculer les prochaines intersections de la balle avec les bordures. Ce faisant, on peut directement calculer les \emph{timelines} pour la balle. À chaque nouvelle intersection, on recommence le calcul.
        
        Les collisions sont gérées sur les vitesses selon l'axe des abscisses ou des ordonnées, selon le type de collision. 

\end{document}